\section{Methodology}

\begin{frame}\frametitle{\secname}\framesubtitle{System Architecture}
\begin{columns}[T,totalwidth=\textwidth]
\column{0.6\textwidth}
\begin{itemize}
    \item<1-> \textbf{LLM Backbone}: Uses pre-trained language models (e.g., Claude, GPT-4) via Marvin framework
    \item<2-> \textbf{Agent-Based Design}: Multi-agent system with specialized agents for different tasks
    \item<3-> \textbf{OSC Communication}: Open Sound Control protocol for real-time DAW control
    \item<4-> \textbf{Modular Tools}: Extensible tool system for different Ableton Live operations
    \item<5-> \textbf{Conversation Management}: Thread-based persistence with SQLite database
\end{itemize}

\column{0.4\textwidth}
\centering
% Placeholder for architecture diagram
\only<1>{\textcolor{gray}{[System Architecture Diagram]}}
\only<2>{\textcolor{gray}{[Agent Structure]}}
\only<3>{\textcolor{gray}{[OSC Layer]}}
\only<4>{\textcolor{gray}{[Tool System]}}
\only<5>{\textcolor{gray}{[Database Schema]}}
\end{columns}
\end{frame}

\begin{frame}\frametitle{\secname}\framesubtitle{Agent Pipeline}
\begin{columns}[T,totalwidth=\textwidth]
\column{0.55\textwidth}
\begin{enumerate}
    \item<1-> \textbf{Input Classification}: Categorize user request (track, clip, device, etc.)
    \item<2-> \textbf{Disambiguation}: Resolve ambiguous references (e.g., "the first track")
    \item<3-> \textbf{Extraction}: Extract structured parameters from natural language
    \item<4-> \textbf{Task Creation}: Generate executable tasks with specific instructions
    \item<5-> \textbf{Execution}: Execute tasks via OSC commands to Ableton Live
\end{enumerate}

\column{0.45\textwidth}
\centering
% Placeholder for pipeline diagram
\only<1>{\textcolor{gray}{[Classification Agent]}}
\only<2>{\textcolor{gray}{[Disambiguation Agent]}}
\only<3>{\textcolor{gray}{[Extraction Agent]}}
\only<4>{\textcolor{gray}{[Task Agent]}}
\only<5>{\textcolor{gray}{[Execution Flow]}}
\end{columns}
\end{frame}

\begin{frame}\frametitle{\secname}\framesubtitle{Key Components}
\begin{itemize}
    \item<1-> \textbf{Classification Agent}: Determines request category (song, track, clip, device, etc.)
    \item<2-> \textbf{Disambiguation Agent}: Resolves ambiguous references using context
    \item<3-> \textbf{Extraction Agent}: Extracts structured data from natural language
    \item<4-> \textbf{Task Agent}: Creates and executes sequences of operations
    \item<5-> \textbf{OSC Tools}: Modular tools for each Ableton Live operation type
    \item<6-> \textbf{Message Formatting}: Filters and formats agent messages for display
\end{itemize}
\end{frame}

\begin{frame}\frametitle{\secname}\framesubtitle{Supported Operations}
\begin{columns}[T,totalwidth=\textwidth]
\column{0.5\textwidth}
\textbf{Song Control}
\begin{itemize}
    \item Playback (play, stop, pause)
    \item Tempo and time signature
    \item Metronome control
    \item Session recording
\end{itemize}

\textbf{Track Management}
\begin{itemize}
    \item Create/delete tracks
    \item Track naming and routing
    \item Volume and pan control
\end{itemize}

\column{0.5\textwidth}
\textbf{Clip Operations}
\begin{itemize}
    \item Create/delete clips
    \item Clip slot management
    \item Launch and stop clips
\end{itemize}

\textbf{Device Control}
\begin{itemize}
    \item Add/remove devices
    \item Parameter adjustment
    \item Device management
\end{itemize}
\end{columns}
\end{frame}

